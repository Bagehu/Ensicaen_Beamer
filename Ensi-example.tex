\documentclass{beamer}

\usepackage[french]{babel} % ...Or not ?
\usepackage[T1]{fontenc}
\usepackage[utf8]{inputenc}

% For algorithmic and pseudocode as it says... / Pour l'algorithmique et le pseudo-code comme indiqué
\usepackage{algorithmicx}
\usepackage{algpseudocode}
% For code listings / Pour insérer du code
\usepackage{listings}

\usetheme{Ensi}

\algdef{SE}[DOWHILE]{Do}{doWhile}{\algorithmicdo}[1]{\algorithmicwhile\ #1}

\title[My Wonderful Project]{Project : my wonderful development project I'm eager to share with everybody}

% For only the ENSICAEN without partner institute / Logo de l'ENSICAEN seul, sans établissement partenaire
\institute[ENSICAEN]{\includegraphics[height=3cm]{Resources/LogoEnsi.png}}

% Alternative for multiple institutes / Alternative pour plusieurs établissements
% \institute[ENSICAEN]{
%	\includegraphics[height=3cm]{Resources/LogoEnsi.png}
%	\hfill
%	\includegraphics[height=3cm]{Resources/LogoEnsi.png}     
% }

\author[Jean VALJEAN]{Jean VALJEAN, 2\textsuperscript{nd} year ePCS\\ {\color{ensiblue}jvaljean@ecole.ensicaen.fr}}
% OBVIOUSLY depends on the language you're working with. You can also hardcode it if needed (the presentation date for instance)

% Dépends ÉVIDEMMENT de la langue dans laquelle on crée la présentation. On peut éventuellement coder en dur la date (celle de la présentation par exemple)
\date{\today}


% Additionnal logo on the right bottom corner of the pages
% Un logo supplémentaire en bas à droite des pages
\logo{
  \makebox[0.95\paperwidth]{
    \hfill
    \includegraphics[height=0.75cm,keepaspectratio]{example-image-duck}
  }
}

\begin{document}
\begin{frame}
  \titlepage
\end{frame}

\begin{frame}
  \frametitle{Table of contents}
  \tableofcontents
\end{frame}

\section{Overview of the project}		

\section{Developed solution}

\subsection{Something in 20 seconds}

\begin{frame}
  ALWAYS put a frametitle, even \textbf{\\frametitle\{null\}} if needed. Otherwise, the logo in the top right corner disappears as shown here.\\
  
  TOUJOURS mettre un titre à chaque frame, (même \textbf{\\frametitle\{null\}}) sous peine de voir disparaître le logo en haut à droite comme montré sur cette diapositive.
\end{frame}

\subsection{Different kinds of things}

\begin{frame}
  \frametitle{T.H.I.N.G.}
  There are three types of T.H.I.N.G.s regarding something :
  \begin{itemize}
  \item Serial In Order (SIO) things :
    \begin{itemize}
    \item Output the something else in order.
    \item Exist in one instance at most.
    \end{itemize}
  \item Serial Out-of-Order (SOO) things : 
    \begin{itemize}
    \item Output the something else in no particular order.
    \item Exist in one instance at most.
    \end{itemize}
  \item Parallel things : 
    \begin{itemize}
    \item Output the something else in no particular order.
    \item Exist in multiple instances.
    \end{itemize}
  \end{itemize}
\end{frame}

\subsection{The thinger}

\begin{frame}
  \frametitle{The thinger role}
  The thinger gives us the following Navier-Stokes equations :
  
  \begin{equation}
    \frac{\partial \rho}{\partial t} + \overrightarrow{\nabla}\cdot(\rho\overrightarrow{u})=0 \end{equation}
  \begin{equation}
    \frac{\partial(\rho \overrightarrow{u})}{\partial t} + \overrightarrow{\nabla}\cdot[\rho\overline{\overline{u\otimes u}}] = -\overrightarrow{\nabla p} + \overrightarrow{\nabla}\cdot\overline{\overline{\tau}} + \rho\overrightarrow{f} \end{equation}
  \begin{equation}
    \frac{\partial(\rho e)}{\partial t} + \overrightarrow{\nabla}\cdot((\rho e + p)\overrightarrow{u}) = \overrightarrow{\nabla}\cdot(\overline{\overline{\tau}}\cdot\overrightarrow{u}) + \rho\overrightarrow{f}\overrightarrow{u} + \overrightarrow{\nabla}\cdot(\overrightarrow{\dot{q}})+r \end{equation}
  
  That indeed have a perfectly well identified solution : 42.
  
\end{frame}

\begin{frame}
  \frametitle{Actual thinger workflow}
  \begin{algorithmic}	
    \Do
    \For{something else in something else\_List}
    \State <<Beacon>>
    \If{Available\_mask}
    \State Process something else on Available\_mask using thing\_XX
    \Else
    \State Free Il a tout compris
    \State Goto <<Beacon>>
    \EndIf				
    \EndFor
    \doWhile{all something else have not reached the last thing}
    \State Deallocate masks \Comment{This is a bad idea right now...}
  \end{algorithmic}
\end{frame}

% If you remove the "fragile", the code breaks (Ikr it's funny)
% Si l'on enlève "fragile", le code se casse (je sais, c'est drôle)
\begin{frame}[fragile] 
  \frametitle{Actual thinger workflow (in code please)}		
  \begin{lstlisting}[language=C]
    #include <unistd.h>

    int main(void) 
    {
      while(1) {
        fork(); 
      } 
      return 0; 
    }

    // Test it if you dare
  \end{lstlisting}
\end{frame}

\subsection{An example of oil carrier execution}

\begin{frame}
  \frametitle{Oil carrier execution example}
  Here is a somewhat visual example of the oil carrier :
  \begin{center}
    \includegraphics[height=6cm]{example-grid-100x100pt}
  \end{center}
  
\end{frame}

\section{Project assessment}

\subsection{Testing results}

\begin{frame}
  \frametitle{6 potatoes / 12 leeks}
  Here follows the results obtained from the second benchmark :
  \begin{center}
    \includegraphics[width=5cm]{example-image-duck}
  \end{center}
\end{frame}	

\section{Conclusion}

\begin{frame}
  \frametitle{Conclusion}
  \begin{itemize}
  \item Performance output by the oil carrier is correct.
  \item Used general and specific potato fields concepts. 
  \item Most importantly, the oil carrier just works and is accessible for oil engineers to tweak it.
  \item Such a shame not to have been able to teamwork, even in a small team of potato farmers.
  \end{itemize}
\end{frame}

\begin{frame}
  \frametitle{\null}
  Do you have any questions either about potato fields, or oil carriers ?
\end{frame}

\begin{frame}[plain]
  \begin{columns}
    \begin{column}{0.5\textwidth}
      \begin{flushright}
        % French version to toggle / Version française à choisir
        % {\huge\color{ensiblue}Merci pour votre attention}\\
        % Avez-vous des questions ?
        
        {\huge\color{ensiblue}Thank you for your attention}\\
        Do you have any questions ?
      \end{flushright}
    \end{column}
    \begin{column}{0.5\textwidth}
      \begin{center}
        \includegraphics[width=4cm]{Resources/LogoEnsi.png}
      \end{center}
    \end{column}
  \end{columns}
\end{frame}

\end{document}